\section{Auswertung}
\label{sec:Auswertung}

\subsection{Matrixrechnung}

Die Matrix $\underline{\underline{A}}$ aus Gleichung \ref{eqn:matrixA} für die Projektionen $I_j$ aus Abbildung \ref{fig:Aufbau2} hat die Form 

\begin{align}
  \underline{\underline{A}} = \left[\begin{matrix}0 & \sqrt{2} & 0 & \sqrt{2} & 0 & 0 & 0 & 0 & 0\\0 & 0 & \sqrt{2} & 0 & \sqrt{2} & 0 & \sqrt{2} & 0 & 0\\0 & 0 & 0 & 0 & 0 & \sqrt{2} & 0 & \sqrt{2} & 0\\1 & 1 & 1 & 0 & 0 & 0 & 0 & 0 & 0\\0 & 0 & 0 & 1 & 1 & 1 & 0 & 0 & 0\\0 & 0 & 0 & 0 & 0 & 0 & 1 & 1 & 1\\0 & \sqrt{2} & 0 & 0 & 0 & \sqrt{2} & 0 & 0 & 0\\\sqrt{2} & 0 & 0 & 0 & \sqrt{2} & 0 & 0 & 0 & \sqrt{2}\\0 & 0 & 0 & \sqrt{2} & 0 & 0 & 0 & \sqrt{2} & 0\\0 & 0 & 1 & 0 & 0 & 1 & 0 & 0 & 1\\0 & 1 & 0 & 0 & 1 & 0 & 0 & 1 & 0\\1 & 0 & 0 & 1 & 0 & 0 & 1 & 0 & 0\end{matrix}\right].
\end{align}

Damit menschliche Rechenfehler vermieden werden, werden alle Matrixrechnungen, auch Inversion und Transposition, in den folgenden Abschnitten mit dem Python-Modul \texttt{sympy} \cite{sympy} bearbeitet. Das Python-Modul \texttt{sympy} ist ein Computeralgebrasystem zum Rechnen mit symbolischen Ausdrücken. 

\subsection{Spektrum der $^{137}Cs$-Quelle}

Für die bestimmung der Prozesse in dem Messpektrum wurden die gemessenen Ereignisse am $NaJ$-Detektor grafisch gegen die Channels dargestellt in Abbildung \ref{fig:leer}. Bei dieser Messung war der erste Würfel jedoch zwischen $\gamma$-Quelle und Detektor. Da dieser Würfel innen leer ist, 
stellt die Messung eine gute Näherung zum Leerlauf dar. 
\begin{figure}[H]
  \centering
  \includegraphics[width = 0.9 \textwidth]{Daten/leerlauf.pdf}
  \caption{Die identifizierten Prozesse in dem Spektrum der Messung der $^{137}Cs$-Quelle. Die Quelle strahlt bei dieser Messung durch die Hauptdiagonale des leeren Würfels. }
  \label{fig:leer}
\end{figure}

In der Abbildung \ref{fig:leer} sind die erkannten Prozesse im Spektrum eingezeichnet. Dabei ist der Photo-Peak deutlich im Channel X zu erkennen. Nach der Literatur liegt dieser Punkt bei der Energie $E_P = 662 \,\si{\kilo\electronvolt}$ \cite{Peak}. 
\subsection{Würfel 1}


\begin{table}[H]
  \centering
  \begin{tabular}{c c c c}
    \toprule
     Projektion &  $t \:/\: \si{\second}$ &                   $N$ &           $I \:/\: \si{\per\second}$ \\
    \midrule
              $I_{4}$ &   $300$ & $49176\pm 306$ & $163.9\pm 1.0$ \\
              $I_{7}$ &   $300$ & $48599\pm 320$ & $162.0\pm 1.1$ \\
              $I_{8}$ &   $300$ & $48599\pm 320$ & $162.0\pm 1.1$ \\
    \bottomrule
  \end{tabular}
  \caption{Die gemessenen Net-Areas des Photo-Peaks und die entsprechende Zählraten des leeren Würfels, welcher nur aus der Aluminiumhülle besteht}
  \label{tab:w1}
\end{table}

\subsection{Würfel 2}

\begin{table}[H]
  \centering
  \begin{tabular}{c c c c}
    \toprule
     Projektion &  $t \:/\: \si{\second}$ &     $N$ &           $I \:/\: \si{\per\second}$ \\
    \midrule
            $I_{  4}$ &   $300$ & $41777 \pm    300$ & $139.3\pm1.0$ \\
            $I_{  5}$ &   $300$ & $43230 \pm    289$ & $144.1\pm1.0$ \\
            $I_{  6}$ &   $300$ & $41772 \pm    287$ & $139.2\pm1.0$ \\
            $I_{  7}$ &   $300$ & $43714 \pm    273$ & $145.7\pm0.9$ \\
            $I_{  8}$ &   $300$ & $42274 \pm    259$ & $140.9\pm0.9$ \\
            $I_{  9}$ &   $300$ & $41326 \pm    297$ & $137.8\pm1.0$ \\
            $I_{ 12}$ &   $300$ & $41886 \pm    297$ & $139.6\pm1.0$ \\
            $I_{ 11}$ &   $300$ & $42662 \pm    288$ & $142.2\pm1.0$ \\
            $I_{ 10}$ &   $300$ & $44080 \pm    280$ & $146.9\pm0.9$ \\
    \bottomrule
    \end{tabular}
  \caption{Die gemessenen Net-Areas des Photo-Peaks und die entsprechende Zählraten des zweiten Würfels. }
  \label{tab:w2}
\end{table}

\subsection{Würfel 3}

\begin{table}[H]
  \centering
  \begin{tabular}{c c c c}
    \toprule
    Projektion &  $t \:/\: \si{\second}$ &     $N$ &           $I \:/\: \si{\per\second}$ \\
    \midrule
             $I_{ 4}$ &   $300$ & $1788 \pm   55$ & $5.96\pm0.18$ \\
             $I_{ 5}$ &   $300$ & $1751 \pm   56$ & $5.84\pm0.19$ \\
             $I_{ 6}$ &   $300$ & $2148 \pm   58$ & $7.16\pm0.19$ \\
             $I_{ 7}$ &   $300$ & $1928 \pm   60$ & $6.43\pm0.20$ \\
             $I_{ 9}$ &   $300$ & $2689 \pm   62$ & $8.96\pm0.21$ \\
             $I_{ 8}$ &   $300$ & $1358 \pm   72$ & $4.53\pm0.24$ \\
             $I_{12}$ &   $300$ & $2145 \pm   60$ & $7.15\pm0.20$ \\
             $I_{11}$ &   $300$ & $1829 \pm   60$ & $6.10\pm0.20$ \\
             $I_{10}$ &   $300$ & $1904 \pm   60$ & $6.35\pm0.20$ \\
    \bottomrule
  \end{tabular}
  \caption{Die gemessenen Net-Areas des Photo-Peaks und die entsprechende Zählraten des dritten Würfels. }
  \label{tab:w3}
\end{table}

\subsection{Würfel 4}


\begin{table}[H]
  \centering
  \begin{tabular}{c c c c}
    \toprule
    Projektion &  $t \:/\: \si{\second}$ &     $N$ &           $I \:/\: \si{\per\second}$ \\
    \midrule
        $I_{  4}$ &   $300$ & $14752 \pm     146$ & $  49.2\pm0.5$ \\
        $I_{  5}$ &   $300$ & $13959 \pm     170$ & $  46.5\pm0.6$ \\
        $I_{  6}$ &   $300$ & $13095 \pm     168$ & $  43.6\pm0.6$ \\
        $I_{  7}$ &   $300$ & $ 9203 \pm     140$ & $  30.7\pm0.5$ \\
        $I_{  8}$ &   $300$ & $ 8382 \pm     107$ & $  27.9\pm0.4$ \\
        $I_{  9}$ &   $300$ & $11430 \pm     194$ & $  38.1\pm0.6$ \\
        $I_{ 12}$ &   $300$ & $41959 \pm     274$ & $ 139.9\pm0.9$ \\
        $I_{ 11}$ &   $300$ & $ 1732 \pm      52$ & $  5.77\pm0.17$ \\
        $I_{ 10}$ &   $300$ & $40324 \pm     279$ & $ 134.4\pm0.9$ \\
        $I_{  1}$ &   $300$ & $ 9735 \pm     161$ & $  32.5\pm0.5$ \\
        $I_{  2}$ &   $300$ & $ 8366 \pm     104$ & $ 27.89\pm0.35$ \\
        $I_{  3}$ &   $300$ & $13498 \pm     170$ & $  45.0\pm0.6$ \\
      \bottomrule
  \end{tabular}
  \caption{Die gemessenen Net-Areas des Photo-Peaks und die entsprechende Zählraten des vierten Würfels. }
  \label{tab:w4}
\end{table}