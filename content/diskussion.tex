\section{Diskussion}
\label{sec:Diskussion}


\subsection{Würfel 2}

\subsection{Würfel 3}

\subsection{Würfel 4}

\subsection{Fehlerquellen}

Es gibt einige Fehlerquellen, die beim Messverfahren beobachtet wurden. Zum einen lässt sich der Würfel nur mit Augenmaß und per Hand in den Strahlengang platzieren. 
Deswegen führt es zu ungenauen Einfallswinkeln und Positionierungen des Würfels im Strahl. In der Realität ist der $\gamma$ Strahl nicht punktförmig, sondern er besitzt eine endlich breite Ausdehnung, 
so dass bei der diagnoalen Messung nicht nur die diagonal angeordneten Würfel vom Strahl getroffen werden, sondern auch die seitlich angrenzenden. \\
Dies kann zu einer Verfälschung der Absorptionskoeffizienten führen. Eine weitere Fehlerquelle Fehlerquelle könnte das ungenaue Justieren des Würfels sein. 
Für eine genauere Justierung, soll am besten die Messvorrichtung verbessert werden und zum einen kann der Strahlengang verkleinert werden, was zu kleineren Intensitäten führen würde und gleichzeitig zu längeren Messzeiten. Es könnte mehr Strahlengänge aufgenommen werden, vor allem die diagonalen Strahlengänge, da sie durch ungenaue Justierung schneller auffallen, als die Strahlengänge, die senkrecht durch den Würfel durchstrahlen. 
